%%%%%%%%%%%%%%%%%%%%%%%%%%%%%%%%%%%%%%%%%%%%%%%%%%%%%%%%%%%%%%%%%%%%%
%                                                                   %
%                             Report                                %
%                                                                   %
%%%%%%%%%%%%%%%%%%%%%%%%%%%%%%%%%%%%%%%%%%%%%%%%%%%%%%%%%%%%%%%%%%%%%

\documentclass[a4paper,12pt,oneside]{report}
\usepackage[english]{babel}

\topmargin 0cm 
\headsep 1cm 
\headheight 0.6cm
\textwidth 14.6cm
\textheight 21.8cm
\evensidemargin 1cm 
\oddsidemargin 1cm
 
 
%%%%%%%%%%%%%%%%%%%%%%%%%%%%%%%%%%%%%%%%%%%%%%%%%%%%%%%%%%%%%%%%%%%%%
%                             Package                               %
%%%%%%%%%%%%%%%%%%%%%%%%%%%%%%%%%%%%%%%%%%%%%%%%%%%%%%%%%%%%%%%%%%%%%

% Symbols
\usepackage{amsmath}                       
\usepackage{textcomp}                      
\usepackage[mathscr]{euscript}

% Figure
\usepackage{epsfig}
\usepackage{graphicx}
\usepackage{subfigure}

% Headers
\usepackage{fancyhdr}

% Encoding characters
\usepackage[ansinew]{inputenc}
\usepackage[OT1]{fontenc}

% Captions
\usepackage[footnotesize,hang,bf]{caption2}

% Link
\usepackage[pdftex]{hyperref} 

%%%%%%%%%%%%%%%%%%%%%%%%%%%%%%%%%%%%%%%%%%%%%%%%%%%%%%%%%%%%%%%%%%%%%
%                            Style                                  %
%%%%%%%%%%%%%%%%%%%%%%%%%%%%%%%%%%%%%%%%%%%%%%%%%%%%%%%%%%%%%%%%%%%%%

% Link in pdf
\hypersetup{%
    bookmarks=true,%
    colorlinks=false,%
    citebordercolor={0 1 0},%
    linkbordercolor={1 0 0},%
    urlbordercolor={0 1 0},%
}       


% Style
\fancyhead[L]{\chaptername \thechapter}
\fancyhead[LO]{\thesection} \fancyhead[RO]{\sectionmark}
\lhead{\slshape Ch. \thechapter}

%%%%%%%%%%%%%%%%%%%%%%%%%%%%%%%%%%%%%%%%%%%%%%%%%%%%%%%%%%%%%%%%%%%%%
%              Configuration and new command                        %
%%%%%%%%%%%%%%%%%%%%%%%%%%%%%%%%%%%%%%%%%%%%%%%%%%%%%%%%%%%%%%%%%%%%%

% New lenght setting
\newlength{\defbaselineskip}
\setlength{\defbaselineskip}{\baselineskip}

% Set \baselineskip as a multiple of \defbaselineskip
\newcommand{\setlinespacing}[1]%
           {\setlength{\baselineskip}{#1 \defbaselineskip}}

% Style of captions images
\renewcommand{\captionfont}{%
	\normalfont \sffamily \slshape \footnotesize
}

% Section instead of Chapter
\addto\captionsitalian{\renewcommand{\chaptername}{Section}}

% Image insert
\newcommand{\imageinsert}[3]{
\begin{center}
\includegraphics[width=#3\columnwidth]{images/#1}
\captionof{figure}{#2}
\end{center}}

% Log source code
\newenvironment{logs}{\begin{flushleft}\begin{ttfamily}\scriptsize\setlinespacing{1}}{\end{ttfamily}\end{flushleft}}


%%%%%%%%%%%%%%%%%%%%%%%%%%%%%%%%%%%%%%%%%%%%%%%%%%%%%%%%%%%%%%%%%%%%%
%                         Document                                  %
%%%%%%%%%%%%%%%%%%%%%%%%%%%%%%%%%%%%%%%%%%%%%%%%%%%%%%%%%%%%%%%%%%%%%


\begin{document}

\thispagestyle{empty}
\enlargethispage{60mm}
\begin{center}
\Large{\textsc{Politecnico di Milano}}\\
\large{V Faculty of Engineering}\\
\large{MSc in Computer Engigneering}\\
\large{School of Industrial Engineering and Information Engineering}\\
\vspace{30mm}
\begin{figure}[h]
\begin{center}
\includegraphics[width=50mm]{images/logo_poli.png}
\end{center}
\end{figure}
\vspace{15mm}

\begin{LARGE}
\textbf{PROJECT RTOS}
\end{LARGE}
\vspace{30mm}

\begin{flushright}
\begin{tabular}{l l }
Report of RTOS: & \\
Omar Scotti & Matr. 798854 \\
Diego Rondelli & Matr. 817108\\
\end{tabular}
\end{flushright}
\vspace{30mm}
{\large{Academic Year 2013-2014}}
\end{center}

\clearpage
\newpage

\setlinespacing{1.5}

\pagenumbering{Roman}

\tableofcontents
\newpage
\clearpage

\listoffigures
\clearpage

\pagestyle{fancy} 
\headsep=40pt 
\lhead{} 
\rhead{\slshape \leftmark} 
\cfoot{\thepage}

\include{abstract}
\pagenumbering{arabic}

\chapter{\textit{Introduction}}

The aim of the project is to prototipe a step counter using prototyping tools, 
as is normaly done for real commercial products.\\
For the project we used a developement board, the STM32F4 Discovery, 
and a real time operative system for embedded system, Miosix, that is developed
at Politecnico di Milano.\\
The project is also been extended with the work of other two groups; this work
of integration bring to the product social wireless and audio feedback functions.

\newpage{}


\chapter{\textit{Tools}}


\section{Board}

For the project developement we used STM32F4 Discovery, made by STMicroelectronics, 
provides integrated hardware pheripheral and a great number of customizable GPIO.\\
To develope the step counter we used:
\begin{itemize}
\item Microcontroller STM32F407VGT6 based on ARM Cortex
M4F architecture; it have 192KB of RAM memory, 1MB of flash memory ed work
at a frequency of 168MHz.
\item ST LIS302DL 3-axis accelerometer with a variable range between \textpm{}2g/\textpm{}8g,
made with MEMS technology.
\item BUS SPI for the comunication between microcontroller and accelerometer.
\end{itemize}


\subsection{Accelerometer ST LIS302DL}

To reveal a step we need to know the user's movement.\\
To do this we used the accelerometer ST LIS302DL, embedded on the board.\\
The accelerometer uses SPI BUS to interface with the microcontroller.\\
\imageinsert{accellerometer.jpg}{Accellerometer}{0.8}
The accelerometer is made with MEMS technology and have different registers, configurable by the user,
to adapt the functions of the peripheral for the application in wich it will be used; 
the main customizable features allow to:
\begin{itemize}
\item enable or disable an axe
\item change the maximum readed acceleration from \textpm{}2g to \textpm{}8g;
Increasing the maximum readed acceleration reduces the sensibility; vice versa, reducing the 
maximum acceleration increases the sensibility.
\end{itemize}


\subsection{SPI BUS}

SPI BUS allows the comunication between microcontroller and accelerometer.
SPI is managed by microcontroller with two internal register,
SPI\_CTRLREG1 and SPI\_CTRLREG2; there is also a third register,
SPI\_SR, that shows the SPI status.\\
Editing this registers the behaviour of the SPI can be adapted to interface with the peripherals connected to it.\\
To ensure the communication between the micro controller and the accelerometer we needed to develop a custom driver
for the SPI. This driver provides two main functionality:
\begin{itemize}
\item read from an address
\item write at an address
\end{itemize}
The read is made by three phases:
\begin{enumerate}
\item An 8-bit burst in sent on SPI; the first bit
identify a multiple operation (in this case the peripheral auto-increment the read address), the second bit identify a read operation,
and the remaining six bits contain the address were the read have to be performed.
\item An 8-bit burst with only 0 is sent on SPI;
In this way we can control the signal on the SPI during the operation.
\item The received burst is saved into the buffer provided by the function caller.
\end{enumerate}
Write is made by three phases:
\begin{enumerate}
\item An 8-bit burst in sent on SPI; the first bit
identify a multiple operation (in this case the peripheral auto-increment the read address), the second bit identify a write operation,
and the remaining six bits contain the address were the write have to be performed.
\item The 8-bit burst that have to be written is sent on SPI
\item The burst sent by the peripheral is read to ensure that the operation went good.
\end{enumerate}

\section{Miosix}

Miosix is a real time operative system released under GNU
Public License, developed at Politecnico di Milano.\\
It offers full compatibility with the board and has full support to 
a C/C++ and Miosix library; this functions facilitated the modular development of the project
and the integration with other groups.


\section{GIT}

The synchronisation between the group members is made using Git,
a distributed version control System, released under GPL licence.\\
To ensure synchronisation between team members and to show the state of the project to 
the tutors we used github, an on-line repository.\\
The branch were the code is published on the tutors' repository is grp13.

\newpage{}

\chapter{\textit{Code}}


\section{Task Division}

Thanks to the functionality offered  by Miosix we were able to develop two thread to manage the task of the project.\\
The found task are:
\begin{enumerate}
\item Pedometer, that reads accelerometer data and identifies user steps .
\item Statistics, that executes computations to calculate distance travelled and activity time 
(divided by walk, run and stop time).
\end{enumerate}
Task division reflects the thread division of the project;
there is also a third thread (main thread) that only creates the other two.\\
In the image below you can see the C++ classes of the project.
\imageinsert{classDiagram.jpg}{Class Diagram}{0.5}

\section{Pedometer}

The task pedometer uses classes LIS302DL, Pedometer and Utility;
The class LIS302DL is responsible for receiving data from the SPI and configuring the accelerometer for operation. \\
Once received, these data are processed by the class Pedometer,
which is also concerned to filter the data and to check whether
the received data correspond to a step. \\
This type of operation is made by the private method stepCounter()
in which is performed the recognition algorithm.
\begin {enumerate}
\item The values read from the accelerometer along x, y and z are filtered,
calculating a delta as the average of the last four values minus
the average of the last sixteen values;
\item Now we have to calculate the acceleration as the root of the sum of squares of the
three delta, previously defined
\item Sets the LIMIT, to support the dynamic threshold, so
there is the possibility to distinguish between a situation of walking
and a situation of running
\item We now have two situations of movement, the first one in which the LIMIT
is exceeded and the second one, that is the pause situation, in which it returns to below the LIMIT.
This allows to distinguish the peak acceleration from possible
noise
\item There is also an additional check to verify that the slope of
the board has not undergone remarkable changes during the step, this is 
to avoid to calculate some oscillations or some movements of the pedometer 
as steps.
\item The step is calculated only in the case where the acceleration is sufficiently high
and the variables x, y and z are within a certain range
\end{enumerate}
\imageinsert{pedometer.jpg}{Pedometer Alghoritm}{0.5}

\section{Statistics}

Statistics task uses Statistics class to compute the statistics 
of the training and Utility class to debugging purpose.\\
At first oldSteps is initialized to 0; every two seconds the task reads the actual step count and compute the steps in last 2 seconds.\\
With this value is possible to make an estimation of the average speed in last two seconds, according to the value in the table.

\imageinsert{statistics.jpg}{Average speed based on number of steps in last 2 seconds}{0.5}

With the average speed is possible to distinguish walk from run, to increment time counter (divided by walk, run and stop time) and travelled distance.


\section{Synchronization}

The two thread works on different variables, so it's not necessary use synchronisation 
mechanism. It's necessary sleep the thread to ensure the correct sampling frequency.\\
Pedometer thread samples accelerometer data at 50Hz, statistics thread samples steps travelled by the user at 0.5Hz. 

\section{Code Size}

In the picture below it's shown the synthetic size of Miosix only, Miosix and the pedometer project and Miosix with the full integrated project.\\

\imageinsert{binaryElf.jpg}{Binary size of the project}{0.7}

The big difference between ElfPedometer and ElfComplete .text part is caused by the audio track saved into the flash memory.

In the picture below it's shown the detailed size of Pedometer project with Miosix

\imageinsert{elf.jpg}{Detailed size of code}{0.7}


\chapter{\textit{Integration}}


We extended our project with other two modules, audio feedback and social wireless.\\
The integration is done performing a merge into branch grp63 with the work of grp13, grp31 and grp63.


\section{Audio Feedback}

To integrate with audio feedback is used the function play\_n\_of\_steps(), called every 50 steps.\\
It is also used the functions victory\_Song() and looser\_Song(); this two functions are called based on the result of the comparison with steps read from another pedometer.\\
All this three function are executed on a new thread to ensure the correct sampling frequency of the pedometer module. 


\section{Social Wireless}

For the integration with social wireless module we expose two functions:
\begin{itemize}
\item int getSteps()
\item void compareSteps(int otherSteps)
\end{itemize}
getSteps() returns actual travelled steps.\\
compareSteps(int otherSteps) has as input the steps number that have to be compared with user's steps and plays a winner or looser sound based on the result of the comparison.\\
Social wireless module uses the function getSteps to read the information that is broadcast, and compareSteps when an information from another board is read.
\end{document}
